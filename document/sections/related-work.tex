\section{Related work} \label{releatedWork}
    
    This section covers an overview of the material that can be seen as the direct foundation for this proposal.
    
        % I believe this sections belongs to the related work
    %- What are state models in TESTAR?
    %How are state models in TESTAR Now.
    
    \subsection{Inferring state models in TESTAR}
    
        In a recent student master graduation project, Inferring state models in TESTAR, \cite{thesisMulders}  the foundation is created to build state models with TESTAR. The result can be found inside the \verb|StateModel| package\footnote{\url{https://github.com/TESTARtool/TESTAR_dev/tree/master/testar/src/nl/ou/testar/StateModel}}
        -> is saved in an OrientDB database - > which gives good query options -> viewer is build to see the model. 
        
        Future work might be adding unit test on the part that will be touched since no code coverage has been found for the package.

        %What is a state 3.3
        
        %\subsubsection{State of the source code}
        
        %Mulders concluded in his thesis \cite{thesisMulders} that 
        


    \subsection{State model difference}
        The \verb|StateModel.Difference| package was added by Ricós\cite{stateDiff} and offers a proof of concept for calculating differences between the state models of two versions. 
        
        Ricós started with a proof of concept for state model difference \footnote{\url{https://github.com/TESTARtool/TESTAR_dev/tree/state_model_difference}}
        give two sut version -> calculates difference in added and removed states.
        
        Let $A$ be a set of states of version 1 of the SUT, let $B$ be a set of states of version 2 of the SUT. The removed states can be written as
        \[A-B = \lbrace x | x \in A \wedge x \notin B \rbrace\]
        the states that are added can be written as
        \[B-A = \lbrace x | x \in B \wedge x \notin A \rbrace\]
        
        Future work can be viewing the SUT en all version as a universe of states to see which states are removed and added, maybe moved (if location is not part of the state)
        
        Future work -> adding unit tests since those are not present -> refactoring of the code since some file do not have a clear Separation of concern. -> one can observation the return and input values are form of \verb|string| (e.g. \verb|string| \verb|Set<String>| etc) -> can lead to primitive obsession -> investigation is needed to solve this. 
