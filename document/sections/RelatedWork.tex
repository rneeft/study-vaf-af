\section{Related work} \label{releatedWork}

This section contains an overview of the material that can be seen as a direct foundation for this proposal.

\subsection{Inferring state models in TESTAR}

In a recent student master graduation project, Inferring state mdoels in TESTAR, \cite{thesisMulders}  the foundation is created to build state models with TESTAR. The result can be found inside the StateModel package \footnote{\url{https://github.com/TESTARtool/TESTAR_dev/tree/master/testar/src/nl/ou/testar/StateModel}}
-> is saved in an OrientDB database - > which gives good query options -> viewer is build to see the model. 

Future work might be adding unit test on the part that will be touched since no code coverage has been found for the package.

\subsubsection{OrientDB database}

OrientDB is \textit{"the first Multi-Model Open Source NoSQL DBMS that brings together the power of graphs and the flexibility of documents into one scalable high-performance operational database."} \cite{orientdb1} 
OrientDB used Vertex and Edge classes. A Vertex is an entity that represents a node in the graph. Edge represents an entity that connects vertices. 

% add illustration about this graph.

\subsection{State model difference}
Ricós started with a proof of concept for state model difference \footnote{\url{https://github.com/TESTARtool/TESTAR\_dev/tree/state_model_difference}}
give two sut version -> calculates difference in added and removed states.

Let $A$ be a set of states of version 1 of the SUT, let $B$ be a set of states of version 2 of the SUT. The removed states can be written as
\[A-B = \lbrace x | x \in A \wedge x \notin B \rbrace\]
the states that are added can be written as
\[B-A = \lbrace x | x \in B \wedge x \notin A \rbrace\]

Future work can be viewing the SUT en all version as a universe of states to see which states are removed and added, maybe moved (if location is not part of the state)

Future work -> adding unit tests since those are not present -> refactoring of the code since some file do not have a clear Separation of concern.