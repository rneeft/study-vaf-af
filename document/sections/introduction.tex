\section{Introduction}
% I don't think I am making a correct point here.
"\textit{Software is everywhere and it’s filled with bugs}".\cite{filledWithBugs}\footnote{Says Marieke Huisman at her inaugural lecture\cite{filledWithBugs}}. The phones in our pockets, pacemakers in our body and guidance systems in space all run on software that is created by people. Most of the time it run without issues but if the software fails, it costs the society a lot of money. A report from Consortium for Information \& Software Quality (CISQ) \cite{herb2020cost} calculated that the cost for poor software quality in 2020 was \$2,08 trillion in the USA alone. When comparing the cost with a country's gross national income (GNI), it is the equivalent to the GNI of Brazil in 2019 \cite{worldbank}\footnote{Brazil had a GNI of \$1,92 trillion in 2019 \cite{worldbank}}. Not only the cost for poor software quality is high also the cost for verification and validation. 

%In addition to the cost of poor software quality, 20\% to 50\% of the cost of a development project is spent on verification and validation \cite{alegroth2016}. 

Due to the new development practices, like \acrfull{ci}, dev processes has speed up -> and time for testing decreased cite here. -> automation is therefore important and necessary.
Continues integration (CI) speeds up the dev process which makes less that for testing

This document is structured as follows. Section \ref{background} describes the background of this proposal and contains information that is available but might not be known by students and readers. Section \ref{releatedWork} contains an overview of the material that can be seen as a direct foundation for this proposal. Section \ref{questions} formulates the research questions for the thesis, what is research method will be used and how the questions are being validated. The last section \ref{planning} will outline the planning for the GA-SE.