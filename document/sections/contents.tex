\section{Introduction}
In the handleiding VAF/AF you can find the parts that need to be addressed in your final thesis. Read that before you start.

This current document is a \LaTeX{} template to get you started.


\section{First Level Headings}

Made with {\tt $\backslash$section}.

\subsection{Second Level Headings}

Made with {\tt $\backslash$subsection}.

\subsubsection{Third and Further Level Headings}

Made with {\tt $\backslash$subsubsection}.


Using more than three levels of headings is highly discouraged.


\section{Latex lists}

You can make lists with automatic numbering using the {\tt enumerate} environment:

\begin{enumerate} 
\item Like this,
\item and like this.
\end{enumerate}

or with bullet points using the {\tt itemize} environment

\begin{itemize} 
\item Like this,
\item and like this.
\end{itemize}

or with words and descriptions 
using the {\tt description} environment

\begin{description}
\item[Word] Definition
\item[Concept] Explanation
\item[Idea] Text
\end{description}

\subsection{Research questions}

Research questions could fit nicely in the \LaTeX{} {\tt description} environment.

\begin{description}
	\item [RQ1] How can we \dots
	\item [RQ2] What is the \dots
	\item [RQ3] What are the consequences of \dots
\end{description}

\section{Acronyms}
You should list all the acronyms you use in the thesis in acronyms.tex and use them as follows:

\begin{enumerate}
\item Use the full definition on the first occurrence, for example \acrfull{gui}
\item Use the short one for all the following occurrences, like \acrshort{gui}.
\end{enumerate}


\section{Footnotes}

Indicate footnotes with a number in the text.\footnote{This is a footnote.}
Footnotes are placed at the bottom of the page on which they appear. 


\section{Figures, Tables and Captions}

\begin{table}[h!tbp]
\begin{tabular}{l | r | r| c}
kolom 1 & kolom 2 & kolom 3 & kolom 4 \\
\hline
zon & maan & ster & meteoor\\
gras & graan & groen & grauw\\
\end{tabular}
\caption{Example table}
\label{table-example}
\end{table}

Table~\ref{table-example} shows how to include a table. Note that the first column is left-justified, the right column is centered, and the other two columns are right-justified (because of the \texttt{\{l | r | r | c\}}). More information: \url{https://en.wikibooks.org/wiki/LaTeX/Tables}. 

\texttt{[h!tb]} means: preferably place the table \emph{h}ere, and if that is not possible, at the \emph{t}op of the page, at the \emph{b}ottom, or on a separate \emph{page}. The same positioning advice can be used in figures. Figure~\ref{fig-example} is an example.

\begin{figure}[h!tbp]
\includegraphics[scale=0.5]{pics/LaTeX.png}
\caption{LaTeX}
\label{fig-example}
\end{figure}



\section{Bibliographic references}




