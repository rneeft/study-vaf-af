\documentclass{ou-report-vaf}

% Dit template is gemaakt door P.J. Molijn in het kader van zijn afstuderen aan de OU in 2014.
% Waarvoor hartelijk dank.
% Minieme maar belangrijke wijzigingen zijn aangebracht door E.M. van Doorn
% Het template is versimpeld door Sylvia Stuurman, 2019.
% Het template is aangepast voor de ba informatica door Harrie Passier, Tanja Vos en Pekka Aho 2020 

\usepackage[acronym]{glossaries}
\makeglossaries
\newacronym{api}{API}{application programming interface}
\newacronym{aut}{AUT}{application under test}
\newacronym{ci}{CI}{continuous integration}
\newacronym{cd}{CD}{continuous delivery}
\newacronym{cr}{CR}{capture \& replay}
\newacronym{efg}{EFG}{event flow graph}
\newacronym{eig}{EIG}{event interaction graph}
\newacronym{fsm}{FSM}{finite state machine}
\newacronym{gui}{GUI}{graphical user interface}
\newacronym{ui}{UI}{user interface}
\newacronym{mbgt}{MBGT}{model based GUI testing}
\newacronym{mbt}{MBT}{model based testing}
\newacronym{roi}{ROI}{return on investment}
\newacronym{sut}{SUT}{system under test}
\newacronym{dbms}{DBMS}{database management system}
\newacronym{fittest}{FITTEST}{Future Internet Testing}
\newacronym{PBI}{PBI}{Product Backlog Item}

\begin{document}

%%%% TITLE PAGE %%%%%%%
\pagenumbering{roman} 
%to prevent that the title page will be referred as page 1, 
%which will give the warning that there is a page 1 twice.

\pagestyle{plain}
\begin{titlepage}
\begin{center}

%% Insert the OU logo at the bottom of the page.
\begin{tikzpicture}[remember picture,overlay]
    \node at (current page.south)[anchor=south,inner sep=0pt]{
        \includegraphics[scale=0.7]{pics/OUlogo}
    };
\end{tikzpicture}

%% Extra white space at the top.
\vspace*{2\bigskipamount}

{\color{red}\Huge\bf How can model diff help the tester finding bugs}
\bigskip

{\large Research proposal}

\bigskip \bigskip
by
\bigskip \bigskip

{\Large\bf Rick Neeft}

\bigskip \bigskip\bigskip \bigskip

\begin{tabular}{lll}
%% Add additional information here, per faculty requirements, e.g
    Student number: & 851829973 \\
    Course code: & \textsc{IM}0502\\
    Thesis committee:
        & Prof. Dr. Tanja E.J. Vos (chairman), & Open University \\
        & Dr. Pekka Aho (supervisor), & Open University \\
        & Fernando Pastor Ricós (supervisor), & Universitat Politècnica de València
\end{tabular}

\end{center}
\end{titlepage}
\pagenumbering{arabic} 
%to prevent that the title page will be referred as page 1, 
%which will give the warning that there is a page 1 twice.

\let\cleardoublepage\clearpage
%%%% END TITLE PAGE %%%%%%%

%This will automatically generate your table of contents
\tableofcontents
\newpage

% Acronyms
\printglossary[type=\acronymtype]
\newpage

%%%% CONTENTS %%%%%%%

\section{Introduction}

History of GUI testing.

What role place GUI testing in the testing spectrum? (compared with unit / integration testing)

What is TESTAR?

What is a test oracle?

Windows Automation API.
\newpage

\section{Related work} \label{releatedWork}

This section contains an overview of the material that can be seen as a direct foundation for this proposal.

\subsection{Inferring state models in TESTAR}

In a recent student master graduation project, Inferring state mdoels in TESTAR, \cite{thesisMulders}  the foundation is created to build state models with TESTAR. The result can be found inside the StateModel package \footnote{\url{https://github.com/TESTARtool/TESTAR_dev/tree/master/testar/src/nl/ou/testar/StateModel}}
-> is saved in an OrientDB database - > which gives good query options -> viewer is build to see the model. 
\subsubsection{OrientDB database}

OrientDB is \textit{"the first Multi-Model Open Source NoSQL DBMS that brings together the power of graphs and the flexibility of documents into one scalable high-performance operational database."} \cite{orientdb1} 
OrientDB used Vertex and Edge classes. A Vertex is an entity that represents a node in the graph. Edge represents an entity that connects vertices. 

% add illustration about this graph.

\subsection{State model difference}
Ricós started with a proof of concept for state model difference \footnote{\url{https://github.com/TESTARtool/TESTAR\_dev/tree/state_model_difference}}
give two sut version -> calculates difference in added and removed states.

Let $A$ be a set of states of version 1 of the SUT, let $B$ be a set of states of version 2 of the SUT. The removed states can be written as
\[A-B = \lbrace x | x \in A \wedge x \notin B \rbrace\]
the states that are added can be written as
\[B-A = \lbrace x | x \in B \wedge x \notin A \rbrace\]


\newpage

\section{Research}

\subsection{Context}
What is the context of the research. 

\subsection{Research questions}

How can model diff help testers finding bugs???
%% specify: "What form of the answer
%% remove ambiguity
%% specify: which factors influences the RQ

- How are changes in state models detected? -> Fernando the video. no paper in TESTAR, Murphy. Image recognition might help in telling difference. 

Could it be possible to use abstract image recognition. make a screenshot of an SUT and then fill in parts, like a Text box as a rectangle, button as a solid block etc. 

% \includegraphics{document/pics/abstract-ui.png}

- What can TESTAR learn from Murphy?

- What other tools are available that uses state model difference to reason about a change? 

- How can difference in the state model become visible inside build-in state model visualisation?

- How does the build-in state model work?

sub question: How to get a useful model for the model diff.
what is a useful model... -> depends on model diff requirements

sub question- How can we tell TESTAR to ignore complete entire widgets for state abstraction?

% discusion points:
%- How can developers of GUI help in creating better models generated by TESTAR?
%- Is it bad to add testing hooks into an application? I hypothesise that adding hooks into code is not bad and can be considered a good practice when creating unit or even integration tests. Is my hypothesis correct and is it a correct hypothesis for GUI testing? I know that Microsoft provides attached properties for XAML based application (AutomationIds). Those properties are then used in code-based GUI testing.

\subsection{Research method}

\subsection{Validity}

\subsection{Planning}
\newpage

%%%%% TEMP TEXT TO TEST LATEX FUNCTIONALLY %%%%%%%%%%%%%
%%% REMOVE BEFORE PUBLISH

\section{Random things to test latex}
With a \acrshort{gui} in a \acrshort{api} you can \acrshort{aut} it with the \acrshort{cr} inside the \acrshort{ci}. To calculate the \acrshort{roi} for the \acrshort{efg} and \acrshort{eig}, it is possible to find the \acrshort{fsm}. \acrshort{mbgt} will be \acrshort{mbt} later used in the \acrshort{sut}

This is an example reference \cite{1} and \cite{2} and \cite{3} and c\cite{4}


%%%% BIBLIOGRAPHY %%%%%%%
% possible styles: apacite, abbrv, acm, alpha, apalike, ieeetr, plain, siam and unsrt.
% alpha and abbrv are most used in computer science
\bibliographystyle{alpha}
\bibliography{bibliography}

\end{document}
